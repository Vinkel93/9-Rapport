\chapter{Nomenclature}

\section*{Acronyms}
	
	\begin{tabular}{|l c l|} \hline
		MP		  	&&	Minimization Problem			\\ \hline
		KVL 		&&  Kirchhoff´s Voltage Law 		\\ \hline
		KCL 		&&  Kirchhoff´s Current Law			\\ \hline
		MPC			&&  Model Predictive Control		\\ \hline
		NRMSE	    &&  Normalized Root Mean Square Error	\\ \hline
		MNGB		&&  Matlab Nonlinear Grey Box	\\ \hline
	\end{tabular}
\section*{Symbols}

\begin{adjustbox}{max height=0.49\textheight}


\begin{tabularx}{\textwidth}{|l|X|l|}\hline
	\textbf{Symbol}		&	\textbf{Description} & \textbf{Unit}	\\\hline
\end{tabularx}
\end{adjustbox}


\newpage
\section*{Glossary of mathematical notation}

This section sums up the mathematical notation and terminology used in this report.

\textbf{Upper and lower bounds of a variable}

\begin{equation}
\underline{x} < x < \overline{x} 
\end{equation}

 Where $x \in {\mathbb{R}} $ and $\overline{x}$ and $\underline{x}$ are the upper and lower bounds, respectively.
 
 \textbf{Intervals}

\begin{equation}
	[a,b] =  \{x \in \mathbb{R}|a\leq x \leq b|\}
\underline{x} < x < \overline{x} 
\end{equation}

 Where $\overline{x}$ and $\underline{x}$ are the upper and lower bounds, respectively.

 \textbf{Vectors and matrices}

Vectors and matrices are noted with bold fonts, such that $\bm{v}$ is a vector:

\begin{equation}
\bm{v} = 
\begin{bmatrix}

		 v_1 	\\
		 v_2 	\\
		 \vdots \\
		 v_n

\end{bmatrix}
\in \pmb{{\mathbb{R}}}^{(n \times 1)}
\end{equation}

and $\bm{M}$ is a matrix:

\begin{equation}
\bm{M} = 
\begin{bmatrix}

		 m_{11} & m_{12} & \hdots & m_{1k}	\\
		 m_{21} & m_{22} & \hdots & m_{2k}	\\
		 \vdots & \vdots & \ddots & \vdots	\\
		 m_{n1} & m_{n2} & \hdots &m_{nk} \\

\end{bmatrix}
\in \pmb{{\mathbb{R}}}^{(n \times k)}
\end{equation}

% Vectorfields are vector-valued functions and noted with greek letters such that:

% \begin{equation}
% \bm{v} = 
% \begin{bmatrix}

% 		 v_1 	\\
% 		 v_2 	\\
% 		 \vdots \\
% 		 v_n

% \end{bmatrix}
% \end{equation}

% $\alpha(\bm{v(t)})$

Continues vector variables are noted with $\bm{v(t)}$ such that:

\begin{equation}
\bm{v(t)} = 
\begin{bmatrix}

		 v_1(t) 	\\
		 v_2(t)	\\
		 \vdots \\
		 v_n(t)

\end{bmatrix}
\in \pmb{{\mathbb{R}}}^{(n \times 1)}
\end{equation}

While discrete vector variables are referred to as sequences and noted with $\bm{v[k]}$, such that:

\begin{equation}
\bm{v[k]} = 
\begin{bmatrix}

		 v_1[k] 	\\
		 v_2[k]	\\
		 \vdots \\
		 v_n[k]

\end{bmatrix}
\in \pmb{{\mathbb{R}}}^{(n \times 1)}
\end{equation}

is a sequence, where $k$ is the time step between two entries.

The pseudo inverse of a matrix is noted with $\bm{{M}^{\dagger}}$.

 \textbf{Small-signal and operating point values}

Small-signals are noted with $\hat{u}$ and the operating point values are noted with $\bar{u}$.

 \textbf{Derivatives}

 The partial derivative of a function is noted with

\begin{equation}
\frac{\partial{f(x,y)}}{\partial{x}}
\end{equation}

% The derivative of a vector by scalar is noted with 

% \begin{equation}
% \frac{\partial{\bm{v}}}{\partial{x}}
% \end{equation}

The derivative of a vector by vector is noted with:

\begin{equation}
\frac{\partial{\bm{v}}}{\partial{\bm{w}}} =
\begin{bmatrix}
    \frac{\partial v_{1}}{\partial w_1} & \frac{\partial v_{1}}{\partial w_2} &  \dots  & \frac{\partial v_{1}}{\partial w_n} \\
    \frac{\partial v_{2}}{\partial w_1} & \frac{\partial v_{2}}{\partial w_2} &  \dots  & \frac{\partial v_{2}}{\partial w_n} \\
    \vdots & \vdots &  \ddots & \vdots \\
    \frac{\partial v_{k}}{\partial w_1} & \frac{\partial v_{k}}{\partial w_2} &  \dots  & \frac{\partial v_{k}}{\partial w_n}
\end{bmatrix}
\end{equation}

If the size of vector $\bm{v}$ and $\bm{w}$ are the same, the resulting matrix is referred to as a Jacobian.

The time derivative of a function is noted with

\begin{equation}
\dot{f} = \frac{d f(t)}{dt}
\end{equation}

\textbf{Vector fields}

Vector fields are introduced, and represent vector valued functions such that the mapping is the following:

\begin{equation}
\alpha(\bm{v}) : \pmb{{\mathbb{R}}}^{(n)} \rightarrow \pmb{{\mathbb{R}}}^{(n)} : [v_1, v_2, \hdots, v_n] \rightarrow [\alpha(v_1), \alpha(v_2),\hdots,\alpha(v_n)]
\end{equation}



