\section{Graph representation}  
\label{GraphTheory}

A graph is a formal mathematical way for representing a network, which can be applicable among others in engineering or scientific context such as in mechanical systems, electrical circuits and hydraulic networks \cite{graph_intro}. 

The modelling of the water distribution network is done with the help of Graph Theory(GT). Each terminal of the network is associated with a node and the components of the system correspond to edges \cite{GraphTheoryCarsten}. 
\subsection*{Incidence matrix} 
\label{IncidenceSection}
The incidence matrix, $\bm{H}$, of a graph with \textit{n} nodes and \textit{e} edges is 
defined by $\bm{H} = [a_{ij}]$. Where the number of rows and columns are defined by the amount of nodes and edges respectively. 
Additionally, the particular node and edge is noted with the indices $i$ and 
$j$.

In case of a hydraulic networks, the edges are directed in order to keep track of the direction of the flows in the system. This results in a directed incidence matrix as described below:

\begin{equation}
\label{DiGraph}
 a_{ij} =
		\left\{
		\begin{array}{ll}
		
		1 			&      \text{if the $j^{th}$ edge is incident out of the $i^{th}$ node}	
\\
		-1                       &     \text{if the $j^{th}$ edge is incident into the $i^{th}$ node}
\\


                0                       &      \text{otherwise}

		\end{array}
		\right.
\end{equation}	

In \appref{IncidenceSection} the corresponding incidence matrix of the system is 
shown. 

\subsection*{Cycle matrix}
\label{CycleSection}
A spanning tree, $T \in \mathcal{G}$ is a subgraph which contains all nodes of $\mathcal{G}$ but has no cycles \cite{GraphModel}. 
In order to obtain the spanning tree, it is necessary to remove an edge from each cycle of the graph. The removed edges are called chords. The number of chords, $l$, is governed by the following expression:

\begin{equation}
  \label{Numberofchords}
  l = e - n +1
\end{equation}

By adding a chord to $T$, a cycle is created which is called a fundamental cycle. A graph is conformed by as many fundamental cycles as there are chords\cite{GraphModel}.  
The set of fundamental cycles correspond to the fundamental cycle matrix $\bm{B}$, such that the number of rows and columns are defined by the amount of chords and edges, respectively. 

The cycle matrix of a directed graph can be expressed with $\bm{B} = [b_{ij}]$ where $i$ and $j$ denote the chords and edges:

\begin{equation}
\label{DiGraphCycle}
 b_{ij} =
		\left\{
		\begin{array}{ll}
		
		1 			&      \text{if the edges $j^{th}$ is in the cycle $i^{th}$ and the directions match}	
\\
		-1                       &     \text{if the edges $j^{th}$ is in the cycle $i^{th}$ and the directions are opposite}
\\

                0                       &      \text{otherwise}

		\end{array}
		\right.
\end{equation}	

In \appref{CycleAppendix} the corresponding cycle matrix of the system is 
shown.

\subsection*{Kirchhoff's Law}
\label{KirchhoffSection}

In the same way as it is described in \secref{HydraulicModel}, the graph of a hydraulic network assigns dual variables to every edge: 
the pressure, $\Delta p_k(t)$, and the flow, $q_k(t)$. These two variables can be set as vectors containing the individual flows through the edges and the pressure drop across them:

% \begin{minipage}{0.45\linewidth}
% \begin{equation}
% p(t) =
% \begin{bmatrix}
%          p_1 \\
% 	p_2 \\ 
% 	\vdots \\
% 	p_e \\
% \end{bmatrix}  \nonumber
% \end{equation}  
% \end{minipage}

% \begin{minipage}{0.45\linewidth}
%  \begin{equation}
% q(t) =
% \begin{bmatrix}
%          q_1 \\
% 	q_2 \\ 
% 	\vdots \\
% 	q_e \\
% \end{bmatrix}
% \end{equation}
% \end{minipage}

\begin{equation}
%\[
\bm{\Delta p(t)} =
\begin{bmatrix}
         \Delta p_c \\
	\Delta p_f \\ 
	\vdots \\
	\Delta p_e \\
\end{bmatrix}
\text{and}\hspace{0.3cm}
\bm{q(t)} =
\begin{bmatrix}
         q_c \\
	q_f \\ 
	\vdots \\
	q_e \\
\end{bmatrix}
%\]
\end{equation}

Where the subscript \textit{c} denotes the chords and the subscript \textit{f$_{\cdots}$e} denotes all the edges in the spanning tree.  
In order to derive a model for the hydraulic network, a set of independent flow variables are identified \cite{TowerModel}. These flow variables have the property that their values can be set independently from other flows in the network and they coincide with the flows through the chords. 
Therefore it is convenient to choose the column indexing of the 
$\pmb{H}$ and $\pmb{B}$ matrix, such as:

% \begin{minipage}{0.45\textwidth}
% \begin{equation}
% H = [H_c \quad H_f]
% \end{equation}   \nonumber
% \end{minipage}

% \begin{minipage}{0.45\textwidth}
% \begin{equation}
% B = [B_c  \quad B_f]
% \end{equation}
% \end{minipage}

\begin{equation}
%\[
\bm{H} = [\bm{H_c} \quad \bm{H_f}]
\hspace{0.1cm} \text{and}\hspace{0.3cm}
\bm{B} = [\bm{B_c}  \quad \bm{B_f}] = [\bm{I} \quad \bm{B_f}]
%\]
\label{HandB}
\end{equation}

\begin{minipage}[t]{0.20\textwidth}
Where\\
\hspace*{8mm} $\bm{H_c} \quad \text{and} \quad \bm{B_c}$ \\
\hspace*{8mm} $\bm{H_f} \quad \text{and} \quad \bm{B_f}$ 
\end{minipage}
\begin{minipage}[t]{0.68\textwidth}
\vspace*{2mm}
\hspace*{8mm} are the matrices corresponding to the chords,\\
\hspace*{8mm} are the matrices corresponding to the spanning tree. 
\end{minipage}

Since the edge variables are governed by elements interconnected in the network, 
they must obey the law of conservation of mass and pressure \cite{GraphModel}. 

Kirchhoff´s Current Law (KCL) states that the net sum of all the flows leaving and entering a node is zero. Formulating this statement in matrix form:

\begin{equation}
  \label{KCL}
  \bm{H} \bm{q(t)} = 0
\end{equation}

Furthermore, regarding Kirchhoff´s Voltage Law (KVL) it is stated that at any time the net sum of the pressure drops in a cycle 
is zero. In terms of matrix form:

\begin{equation}
 \label{KVL} 
 \pmb{B}  \pmb{\Delta p (t)} = 0
\end{equation}

where the fundamental loops have a reference direction given by the direction of the 
chords. 





 