\subsection{Water Tower} 
\label{WaterTankModel}

Water towers are used to maintain the correct pressure in different systems, ensure reliability and to improve the optimality of the water supply. The WT plays a determinative role in the control. Therefore it is dynamic model must be derived. 

Similarly to the modelling of the other components, the relation between the two dual variables, pressure difference and flow is derived. The structure of the WT is illustrated in \figref{fig:watertower_sketch}.

%To do so, Bernoulli´s principle is applied between two points (see figure...).

%tikz of the water tower
\begin{figure}[H]
\centering

%Old one: 
%\begin{tikzpicture} [scale=0.9,transform shape]
%  \fill[blue!30]
%        decorate[ragged border]{
%        (0,2) -- (3.5,2)
%        }
%        -- (3.5,-0.5) -- (2,-0.5) -- (2,-1.5) -- (1.5,-1.5) -- (1.5,-0.5) --(1.5,-0.5) -- (0,-0.5) -- cycle;
%
%  \draw [-,line width=1.1pt]  (0,3.5) -- (0,2.5) -- (0,-0.5) -- (1.5,-0.5) -- (1.5,-0.5) -- (1.5,-1.5);
%  \draw [-,line width=1.1pt] (3.5,3.5) --  (3.5,-0.5) -- (2,-0.5) -- (2,-1.5);
%  \draw[|-|] (-0.5,-0.5) --
%        node[fill=white,font=\normalsize,inner ysep=2pt,inner
%                xsep=0]{$h_1$}(-0.5,2);
%                
%  \draw[|-|] (3.5,4) --
%        node[fill=white,font=\normalsize,inner ysep=2pt,inner
%                xsep=0]{$D$}(0,4);
% 
% \draw[red,fill=black] (1.75,-0.5) circle (.3ex);
%  \draw[red,fill=black] (1.75,2) circle (.3ex);
%
%\node at (2,2) {\normalsize{$p_1$}};
%\node at (2,-0.34) {\normalsize{$p_2$}};
%
%\draw [-latex](1.75,3.0) -- (1.75,2.29);
%\node at (2,2.7) {\normalsize{$q_1$}};
%\node at (1.9,-1.67) {\normalsize{$q_2$}};
%\draw [-latex](1.75,-0.7) -- (1.75,-1.5);
%
%%\draw [dashdotdotted](1.75,3.75) -- (1.75,-2.0);
%
%
%\end{tikzpicture}%


\begin{tikzpicture} [scale=0.8,transform shape]
  \fill[cyan!30]
        decorate[ragged border]{
        (0,2) -- (3.5,2)
        }
        -- (3.5,-0.5) -- (2,-0.5) -- (2,-1.5) -- (1.5,-1.5) -- (1.5,-0.5) --(1.5,-0.5) -- (0,-0.5) -- cycle;

  \draw [-,line width=1.1pt]  (0,3.5) -- (0,2.5) -- (0,-0.5) -- (1.5,-0.5) -- (1.5,-0.5) -- (1.5,-1.5);
  \draw [-,line width=1.1pt] (3.5,3.5) --  (3.5,-0.5) -- (2,-0.5) -- (2,-1.5) ;
  \draw[|-|] (-0.5,-0.5) --
        node[fill=white,font=\normalsize,inner ysep=2pt,inner
                xsep=0]{$h$}(-0.5,2);
                
  \draw[|-|] (3.5,4) --
        node[fill=white,font=\normalsize,inner ysep=2pt,inner
                xsep=0]{$D$}(0,4);
 


\node at (2.5,2.1) {\normalsize{$p_a$}};
\node at (2.5,-0.3) {\normalsize{$p$}};


\node at (1.9,-1.9) {\normalsize{$q$}};
\draw [-latex](1.75,-2.1) -- (1.75,-1.2);



\end{tikzpicture}%
 
\caption{Sketch of the open water tower.}
\label{fig:watertower_sketch}
\end{figure}

In \figref{fig:watertower_sketch}, $p_a$ represents the pressure at the surface of the water, thus it always describes the atmospheric value. The variable $p$ is used to describe the pressure value on the bottom of the tank. \\
The rate of change of the fluid volume in the WT is proportional to the volumetric flow at which water enters or leaves the tank. 

\begin{equation}
  q = \frac{dV_t}{dt} = A_{wt} \frac{dh}{dt}
  \label{Flowequation}
\end{equation}

\begin{minipage}[t]{0.20\textwidth}
Where\\
\hspace*{8mm} $h$ \\
\hspace*{8mm} $V_t$ \\
\hspace*{8mm} $A_{wt}$ \\
\hspace*{8mm} \\
and \hspace*{0.7mm} $q$ \\
\end{minipage}
\begin{minipage}[t]{0.68\textwidth}
\vspace*{2mm}
is the height of the fluid in the WT,\\
is the volume of the WT,\\
is the cross section of the WT which is assumed to be constant for $y \in [0,h]$,\\
is the volumetric flow.
\end{minipage}
\begin{minipage}[t]{0.10\textwidth}
\vspace*{2mm}
\textcolor{White}{te}$\unit{m}$\\
\textcolor{White}{te}$\unit{m^3}$\\
\textcolor{White}{te}$\unit{m^2}$\\
\textcolor{White}{te}\\
\textcolor{White}{te}$\unit{\frac{m^3}{s}}$
\end{minipage}

The force on the bottom of the WT is due to the weight of water. According to Newton's second law: 

\begin{equation}
  F = m_wg = \rho g V_t
  \label{Newton_WT}
\end{equation}

  \begin{minipage}[t]{0.20\textwidth}
Where\\
\hspace*{8mm} $\rho$ 
\end{minipage}
\begin{minipage}[t]{0.68\textwidth}
\vspace*{2mm}
is the density of water.
\end{minipage}
\begin{minipage}[t]{0.10\textwidth}
\vspace*{2mm}
\textcolor{White}{te}$\unit{\frac{kg}{m^3}}$
\end{minipage}


\eqref{Newton_WT} can be rewritten in terms of pressure such as: 

\begin{equation}
  \frac{F}{A_{wt}} = \rho g h = p - p_a = \Delta p
  \label{Pressuredifference}
\end{equation}

The total pressure on the bottom of the WT is a result of the pressure difference, $p$, and the atmospheric pressure, $p_a$. However, the model is derived in such a way that the atmospheric pressure is set to zero. Therefore, if the water is assumed to be incompressible, density does not change with pressure and \eqref{Flowequation} can be written as: 

\begin{equation}
q = \frac{dV}{dt} = \frac{A_{wt}}{\rho g} \frac{d}{dt} \Delta p = C_H \Delta \dot{p}
  \label{FlowConservation}
\end{equation}

\begin{minipage}[t]{0.20\textwidth}
Where\\
\hspace*{8mm} $C_H$ 
\end{minipage}
\begin{minipage}[t]{0.68\textwidth}
\vspace*{2mm}
is the hydraulic capacitance.
\end{minipage}
\begin{minipage}[t]{0.10\textwidth}
\vspace*{2mm}
\textcolor{White}{te}$\unit{\frac{m^3}{Pa}}$
\end{minipage}

This equation shows proportionality between pressure and the volume of water, which is the defining characteristic of a fluid capacitor. When the fluid capacitance is large, corresponding to a tower with a large area, a large increase in volume is accompanied by a small increase in pressure. 

The model of the WT is described by a first order differential equation, consisting of the first time derivative of the pressure drop. The final expression is shown in \eqref{CompleteWTModel}:

\begin{equation}
  \label{CompleteWTModel}
  \Delta \dot{p}_{wt,k} = \frac{1}{C_{H,k}} q_k 
\end{equation}

Although \eqref{CompleteWTModel} includes indexing for the pressure drops across the WTs, it is worth mentioning that the water distribution network in this project consists of only one WT.


%---------------------------------------------------

%OLD VERSION:
%
%\subsection{Water Tower} 
%\label{WaterTankModel}
%
%Water towers are used to maintain the correct pressure level in the system, ensure reliability and to improve the optimality of the water supply. The WT plays a determinative role in the control of flow, therefore its dynamic model has to be derived. 
%
%Similarly to the modelling of the other components, the relation between the two dual variables, pressure difference and flow is  derived. The structure of the WT is illustrated in \figref{fig:watertower_sketch}.
%
%%To do so, Bernoulli´s principle is applied between two points (see figure...).
%
%%tikz of the water tower
%\begin{figure}[H]
%\centering
%
%Old one: 
%\begin{tikzpicture} [scale=0.9,transform shape]
%  \fill[blue!30]
%        decorate[ragged border]{
%        (0,2) -- (3.5,2)
%        }
%        -- (3.5,-0.5) -- (2,-0.5) -- (2,-1.5) -- (1.5,-1.5) -- (1.5,-0.5) --(1.5,-0.5) -- (0,-0.5) -- cycle;
%
%  \draw [-,line width=1.1pt]  (0,3.5) -- (0,2.5) -- (0,-0.5) -- (1.5,-0.5) -- (1.5,-0.5) -- (1.5,-1.5);
%  \draw [-,line width=1.1pt] (3.5,3.5) --  (3.5,-0.5) -- (2,-0.5) -- (2,-1.5);
%  \draw[|-|] (-0.5,-0.5) --
%        node[fill=white,font=\normalsize,inner ysep=2pt,inner
%                xsep=0]{$h_1$}(-0.5,2);
%                
%  \draw[|-|] (3.5,4) --
%        node[fill=white,font=\normalsize,inner ysep=2pt,inner
%                xsep=0]{$D$}(0,4);
% 
% \draw[red,fill=black] (1.75,-0.5) circle (.3ex);
%  \draw[red,fill=black] (1.75,2) circle (.3ex);
%
%\node at (2,2) {\normalsize{$p_1$}};
%\node at (2,-0.34) {\normalsize{$p_2$}};
%
%\draw [-latex](1.75,3.0) -- (1.75,2.29);
%\node at (2,2.7) {\normalsize{$q_1$}};
%\node at (1.9,-1.67) {\normalsize{$q_2$}};
%\draw [-latex](1.75,-0.7) -- (1.75,-1.5);
%
%%\draw [dashdotdotted](1.75,3.75) -- (1.75,-2.0);
%
%
%\end{tikzpicture}%


\begin{tikzpicture} [scale=0.8,transform shape]
  \fill[cyan!30]
        decorate[ragged border]{
        (0,2) -- (3.5,2)
        }
        -- (3.5,-0.5) -- (2,-0.5) -- (2,-1.5) -- (1.5,-1.5) -- (1.5,-0.5) --(1.5,-0.5) -- (0,-0.5) -- cycle;

  \draw [-,line width=1.1pt]  (0,3.5) -- (0,2.5) -- (0,-0.5) -- (1.5,-0.5) -- (1.5,-0.5) -- (1.5,-1.5);
  \draw [-,line width=1.1pt] (3.5,3.5) --  (3.5,-0.5) -- (2,-0.5) -- (2,-1.5) ;
  \draw[|-|] (-0.5,-0.5) --
        node[fill=white,font=\normalsize,inner ysep=2pt,inner
                xsep=0]{$h$}(-0.5,2);
                
  \draw[|-|] (3.5,4) --
        node[fill=white,font=\normalsize,inner ysep=2pt,inner
                xsep=0]{$D$}(0,4);
 


\node at (2.5,2.1) {\normalsize{$p_a$}};
\node at (2.5,-0.3) {\normalsize{$p$}};


\node at (1.9,-1.9) {\normalsize{$q$}};
\draw [-latex](1.75,-2.1) -- (1.75,-1.2);



\end{tikzpicture}%
 
%\caption{Sketch of the open water tank}
%\label{fig:watertower_sketch}
%\end{figure}
%
%Two reference points are selected which represent the pressure at specific heights inside the tank. $p_a$ represents the pressure on the surface of the water level, therefore it is the atmospheric pressure at all time.The pressure in point $p_2$ equals to the pressure value on the bottom of the tank. In order to get an expression for pressure differences in the function of the water level $h_1$, Bernoulli's law is applied as it is shown in \eqref{bernoulli}: 
%
%\begin{equation}
%  \label{bernoulli}
%  h_1\rho g + p_1 + \frac{1}{2}\rho {v_1}^2 = h_2\rho g + p_2 + \frac{1}{2}\rho {v_2}^2
%\end{equation}
%
%\begin{minipage}[t]{0.20\textwidth}
%Where\\
%\hspace*{8mm} $h_i$ \\
%\hspace*{8mm} $\rho$ \\
%\hspace*{8mm} $p_i$ \\
%\hspace*{8mm} $v_i$ \\
%
%\end{minipage}
%\begin{minipage}[t]{0.68\textwidth}
%\vspace*{2mm}
%is the elevation of the points represented\\
%is the water density\\
%is the pressure at a chosen point\\
%is the velocity of the water level at a chosen point
%\end{minipage}
%\begin{minipage}[t]{0.10\textwidth}
%\vspace*{2mm}
%\textcolor{White}{te}$\unit{m}$\\
%\textcolor{White}{te}$\unit{\frac{kg}{m^3}}$\\
%\textcolor{White}{te}$\unit{Pa}$\\
%\textcolor{White}{te}$\unit{\frac{m}{s}}$
%\end{minipage}
%
%
%%The pressure at point $1$ ($p_1$) is considered to be equal to the atmospherical pressure 
%%so it is set to be $0$. 
%The height of reference point $1$ changes according to the water level, therefore its position is in the function of $h1$. 
%In \eqref{bernoulli}, the reference point on the bottom of the tank remains fixed.
%%As a reference point in \eqref{bernoulli}, the bottom of the tank is chosen.
%Therefore the height at this point($h_2$) is set to 0. 
%\\
%The flow rates are equal in both point $1$ 
%and point $2$, so are the velocities of the water. \\
%Due to the these considerations \eqref{bernoulli} simplifies to: 
%
%\begin{equation}
%  \Delta p = p_2 - p_1 = \rho g h_1
%  \label{Pressuredifference}
%\end{equation}
%
%In \eqref{Pressuredifference} the gravity and the density are constants, however the height of the water level changes proportional to the pressure difference. The total pressure on the inside of the bottom tank wall is a result of the pressure due to the water $p_2$, plus the
%atmospheric pressure $p_1$. However the model is derived according to the pressure difference between inside and outside, therefore $p_1$ atmospheric pressure is set to zero. 
%\\
%To derive the equation that describes the change of the height according to 
%time, the law of conservation mass is applied. This law can be specifically used for the flow in the water tower, describing the difference between the inlet and outlet flow to be equal to the velocity by which the tank volume increases or decreases. 
%
%\begin{equation}
%  q_{2} - q_{1} = \frac{d}{dt}V_t = \frac{d}{dt}Ah_1 = A \frac{d}{dt}h_1 = \frac{\pi}{4} D^2 \frac{d}{dt} h_1
%  \label{FlowConservation}
%\end{equation}
%
%
%\begin{minipage}[t]{0.20\textwidth}
%Where\\
%\hspace*{8mm} $q_i$ \\
%\hspace*{8mm} $V_t$ \\
%\hspace*{8mm} $D$ 
%\end{minipage}
%\begin{minipage}[t]{0.68\textwidth}
%\vspace*{2mm}
%is the inlet and outlet flow\\
%is the volume of the water in the tower\\
%is the diameter of the tank 
%\end{minipage}
%\begin{minipage}[t]{0.10\textwidth}
%\vspace*{2mm}
%\textcolor{White}{te}$\unit{\frac{kg}{m^3}}$
%\textcolor{White}{te}$\unit{m^3}$
%\textcolor{White}{te}$\unit{m}$
%\end{minipage}
%
%In \eqref{FlowConservation}, $q_11$ is zero since only $q_2$ is taken into consideration as a two-directional in- or outlet flow. 
%\\
%By integrating \eqref{FlowConservation} according to time the following yields:
%
%\begin{equation}
% h_1 = \int dh_1 = \int \frac{1}{A} q_2 dt
%  \label{integheight}
%\end{equation}
%
%Plugging \eqref{integheight} into \eqref{Pressuredifference}, the following overall expression can be given between pressure and flow: 
%
%\begin{equation}
%\label{WTequation}
%  p_2 =  \frac{\rho g}{A}  \int   q_2 dt = \frac{1}{C_H} \int   q_2 dt
%\end{equation}
%
%\begin{minipage}[t]{0.20\textwidth}
%Where\\
%\hspace*{8mm} $C_H$ 
%\end{minipage}
%\begin{minipage}[t]{0.68\textwidth}
%\vspace*{2mm}
%is the hydraulic capacitance
%\end{minipage}
%\begin{minipage}[t]{0.10\textwidth}
%\vspace*{2mm}
%\textcolor{White}{te}$\unit{\frac{m^3}{N/m^2}}$
%\end{minipage}
%
%In \eqref{WTequation}, the time integral of the flow gives back the volume of the water in the water tower. This equation shows proportionality between pressure and the volume of water which is exactly the defining characteristic of a fluid capacitor. When the fluid capacitance $C_H$ is large, corresponding to a tank with a large area, a large increase in volume is accompanied by a small increase in pressure. 
%
%An analogy can be made between an electronic circuit and the hydraulic system, where the WT acts as a capacitor.  Deriving the relationship between the voltage across the capacitor and the 
%charge of the capacitor:
%
%\begin{equation}
%  U = \frac{1}{C} \int I dt
%  \label{ElecCircuirt}
%\end{equation}
%
%\begin{minipage}[t]{0.20\textwidth}
%Where\\
%\hspace*{8mm} $U$ \\
%\hspace*{8mm} $C$ 
%\end{minipage}
%\begin{minipage}[t]{0.68\textwidth}
%\vspace*{2mm}
%is the voltage\\
%is the capacitance 
%\end{minipage}
%\begin{minipage}[t]{0.10\textwidth}
%\vspace*{2mm}
%\textcolor{White}{te}$\unit{V}$
%\textcolor{White}{te}$\unit{F}$
%\end{minipage}
%
%In the \eqref{WTequation} the volume flow rate ($q$) is equivalent to the current ($I$) in a 
%circuit and the constant term \big($\frac{A}{\rho g}$\big) is equivalent to the capacitance of a capacitor ($C$). The voltage drop is analogous to the pressure drop in the water system.
%
