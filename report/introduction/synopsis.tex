% This project covers the modelling and predictive control of a water distribution network with the aim of minimizing energy and economic cost. 
% \newline
% At first the non-linear model of the components and the dynamics of the system are modelled based on a graph-based approach which leads to a state space representation of the whole network. Then system identification is carried out due to the uncertain parameters of the pipe components.
% \newline
% A model predictive controller is applied to the linearized model of the water distribution system extended with an elevation reservoir. The controller follows certain constraints to maintain consumer pressure-desire in two pressure management areas and to optimize the use of water tower such that the cost of pumping effort is minimized. 
% The controller is implemented in a cascade system along with PI controllers and is based on the model of the network, the cost of electricity and the characteristics of end-user water usage. 
% %\newline
% %Implementation carried out ...
% \newline
% The results show that ...



This project covers the modelling and predictive control of a water distribution network with the aim of minimizing economic cost. The system consists of pipes, valves, pumps and additionally a water tower.\\
At first a non-linear model of the components and the dynamics of the system is modelled with a graph-based approach.\\ 
System identification is carried out on the non-linear model due to the uncertainties in the network. The non-linear model is however linearized and a new parameter estimation is carried out with more acceptable results. \\
A model predictive controller is designed to minimize the running cost of the linearized model subject to constraints defined by measurements on a real-world test setup. Furthermore, PI controllers are designed to control the system with the reference set by the MPC. \\
Verification of the controller is attempted by simulation.\\
At the end, a discussion reflects on the results of the simulation and a conclusion sums up the possibilities of further improvements. 





% This project covers the modelling and predictive control of a water distribution network with the aim of minimizing energy and economic cost.
% \newline
% At first a non-linear model of the components and the dynamics of the system are modelled based on a graph-based approach which leads to a state space representation of the whole network. System identification is carried out on the non-linear model due to the uncertain parameters of the pipe components, which showed bad results. The non-linear model is linearized and a new parameter estimations is done, which shows better results.\\

% A model predictive controller is designed to minimize the running cost of the linearized model while maintaining pressure, upper and lower bound constraints for the system.\\
% Furthermore, four PI controllers are designed to controller the pumps in the system. Tow of which are to make a constant pressure lift where the last two are to follow the reference set by the model predictive controller. \\

% Simulations of the model predictive controller without constraints shows 


% %\newline
% %Implementation carried out ...
% \newline
% The results show that ...