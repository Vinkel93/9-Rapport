\chapter{Requirements and Constraints}
\label{Requirements_and_constraints}

Adding a WT to an existing water distribution network introduces new requirements and constraints on the system. 
%\todo{outdated notation}

As mentioned in \secref{introduction}, a minimum pressure must be maintained in the PMAs for the end-users. Furthermore, the pressure can not exceed a maximum level as this might increase the possibility of water leakage and increase the wear on the pipes. The system described in \secref{system_overview}, is designed to operate at a pressure around 0.1 bar, relative to the environment \cite{master_aau}. For the purpose of this project the interval for which the CP pressure should be within, is chosen to be between $0.08 < \bm{y} < 0.18$ [Bar]. Where $\bm{y}$ is the pressure at the CP's which for PMA 1 is chosen as node 10 and for PMA 2 is chosen as node 15. The upper limit is set high to allow that the pumps can provide a lot of pressure to fill the WT when the price is low. 

Another important aspect when implementing a WT is water quality. If the water is stored in the WT for too long, the quality will decrease due to decreasing oxygen level. Thus a requirement for water quality has to be formulated. As described in \secref{system_overview}, the WT has one combined input-output connection. Therefore a requirement for the flow is hard to formulate as the direction can change dependent on the usage. This could result in a flow based constraint being fulfilled by rapidly changing flow direction without actually replacing any significant water volume in the tower. Instead, a requirement for how often the content of the WT should be exchanged per time unit is discussed. It is important however to point out that formulating a constraint on the WT flow is over the limit of this project, therefore only discussion is considered. 
Minimum requirement for volume exchange can be chosen as 30\% of the volume of $V_t$ per day. This can be written as $\bar{q}_{wt} > 0.3\cdot V_t \: \big[\frac{m^3}{day}\big]$. As stated in \secref{system_overview} $V_t = 200 \:L$ so therefore $\bar{q}_{wt} > 0.06 \: \big[\frac{m^3}{day}\big]$.

These constraints, due to the scope of the project, are reduced and thus only contain a requirement for the pressure at each CP and a requirement to minimize the power consumption of the system. Thus, the requirement regarding water exchange in the WT is not investigated any further in this project. 
Therefore the following requirements and constraints are considered:

\begin{itemize}
	\item Constraint on the output, $0.08 < \bm{y} < 0.18 \:[\text{bar}]$
%
	\item Constraint on the WT, $0.055 < \Delta p_{wt} < 0.127  \:[\text{bar}]$
%
	\item Constraint on the inputs, $ 0.05 < \bm{u} < 0.95  \:[\text{bar}]$
%
	\item Minimizing the total cost of running the system
\end{itemize}
 